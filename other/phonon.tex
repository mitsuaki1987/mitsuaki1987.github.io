\documentclass{article}
\usepackage{amsmath}
\usepackage[margin=3cm]{geometry}
\begin{document}

\title{Phonon}
\author{Mitsuaki Kawamura}
\maketitle


\section{Single harmonic Oscillator}

The hamiltonian operator is
\begin{align}
    \hat{H} = \frac{\hat{p}^2}{2 M} + \frac{C \hat{u}^2}{2},
\end{align}
where $\hat{u}$ and $\hat{p}$ are the position and the momentum operator.
These operators satisfies the following commutation relation:
\begin{align}
    [\hat{u}, \hat{p}] \equiv \hat{u} \hat{p} - \hat{p} \hat{u} = i
\end{align}
Then we introduce the creation- and anihiration- operator ($\hat{b}^\dagger$ and $\hat{b}$)
as follows
\begin{align}
    \hat{u} &\equiv \frac{1}{(4 M C)^{1/4}}(\hat{b} + \hat{b}^\dagger),
    \\
    \hat{p} &\equiv \left(\frac{M C}{4}\right)^{1/4}(-i \hat{b} + i \hat{b}^\dagger).
\end{align}
The commutation relation of $\hat{b}^\dagger$ and $\hat{b}$
\begin{align}
    [\hat{b}, \hat{b}] = 0, [\hat{b}, \hat{b}^\dagger] = 1
\end{align}
lead to the orignal commutation relation of $\hat{u}$ and $\hat{p}$ as
\begin{align}
    [\hat{u}, \hat{p}] &= \frac{i}{2}(-[\hat{b}, \hat{b}] + [\hat{b}, \hat{b}^\dagger]
    -[\hat{b}^\dagger, \hat{b}] + [\hat{b}^\dagger, \hat{b}^\dagger])
    =i,
\end{align}
and the hamiltonian becomes
\begin{align}
    \hat{H} &= \frac{1}{4}\left( 
        \frac{(M C)^{1/2}}{M} (-i \hat{b} + i \hat{b}^\dagger)(-i \hat{b} + i \hat{b}^\dagger)
        + \frac{C}{(M C)^{1/2}}(\hat{b}+\hat{b}^\dagger)(\hat{b}+\hat{b}^\dagger)
        \right)
        \nonumber \\
    &=\frac{1}{2} \left(\frac{C}{M}\right)^{1/2} (\hat{b} \hat{b}^\dagger + \hat{b}^\dagger \hat{b})
    = \omega \left( \hat{b}^\dagger \hat{b} + \frac{1}{2}\right),
\end{align}
where $\omega\equiv(C/M)^{1/2}$

\section{Coupled Harmonic Oscillator}

The hamiltonian operator is
\begin{align}
    \hat{H} = \frac{1}{2}\sum_{s}\frac{\hat{p}_s^2}{M_s} + \frac{1}{2}\sum_{s s'} \hat{u}_s  C_{s s'} \hat{u}_{s'},
\end{align}
where $\hat{u}_s$ and $\hat{p}_s$ are the position and the momentum operator.
These operators satisfies the following commutation relation:
\begin{align}
    [\hat{u}_s, \hat{p}_{s'}] \equiv i \delta_{s s'}.
\end{align}
Then, we introduce the creation- and anihiration- operator ($\hat{b}_\nu^\dagger$ and $\hat{b}_\nu$)
as follows
\begin{align}
    \hat{u}_s &\equiv \sum_{\nu} \frac{v_{s \nu}}{(M_s \omega_\nu)^{1/2}} (\hat{b}_\nu + \hat{b}_\nu^\dagger),
    \\
    \hat{p}_s &\equiv \sum_{\nu} (M_s \omega_\nu)^{1/2} v_{s \nu}(-i \hat{b}_\nu + i \hat{b}_\nu^\dagger),
\end{align}
where $v_{s \nu}$ and $\omega_\nu^2$ are eigenvectors and eigenvalues of rescaled force constant as
\begin{align}
    \sum_{s'} \frac{C_{s s'}}{(M_s M_{s'})^{1/2}} v_{s' \nu} = \omega_\nu^2 v_{s \nu}
\end{align}
The commutation relation of $\hat{b}_\nu^\dagger$ and $\hat{b}_\nu$
\begin{align}
    [\hat{b}_\nu, \hat{b}_{\nu'}] = 0, [\hat{b}_\nu, \hat{b}_{\nu'}^\dagger] = \delta_{\nu \nu'}
\end{align}
lead to the orignal commutation relation of $\hat{u}_s$ and $\hat{p}_s$ as
\begin{align}
    [\hat{u}_s, \hat{p}_{s'}] &= \sum_{\nu \nu'} v_{s \nu} v_{s' \nu'}
    \frac{i}{2}(-[\hat{b}_\nu, \hat{b}_{\nu'}] + [\hat{b}_\nu, \hat{b}_{\nu'}^\dagger]
    -[\hat{b}_\nu^\dagger, \hat{b}_{\nu'}] + [\hat{b}_\nu^\dagger, \hat{b}_{\nu'}^\dagger])
    = i \sum_\nu v_{s \nu} v_{s' \nu} = i \delta_{s s'},
\end{align}
and the hamiltonian becomes
\begin{align}
    \hat{H} &= \frac{1}{2} \sum_{\nu \nu'}
        (\omega_\nu \omega_{\nu'})^{1/2} 
        (-i \hat{b}_\nu + i \hat{b}_\nu^\dagger)(-i \hat{b}_{\nu'} + i \hat{b}_{\nu'}^\dagger)
         \sum_{s}\frac{M_s }{M_s} v_{s \nu} v_{s \nu'}
         \nonumber \\
         &+\frac{1}{2} \sum_{\nu \nu'}
         (\hat{b}_\nu + \hat{b}_\nu^\dagger)(\hat{b}_{\nu'} + \hat{b}_{\nu'}^\dagger)
         \frac{1}{(\omega_\nu \omega_{\nu'})^{1/2}}
         \sum_{s s'} v_{s \nu} \frac{C_{s s'}}{(M_s M_{s'})^{1/2}} v_{s' \nu'} 
         \nonumber \\
         &= \frac{1}{2} \sum_{\nu \nu'}
         (\omega_\nu \omega_{\nu'})^{1/2} 
         (-\hat{b}_\nu\hat{b}_{\nu'} + \hat{b}_\nu\hat{b}_{\nu'}^\dagger + \hat{b}_\nu^\dagger\hat{b}_{\nu'} - \hat{b}_\nu^\dagger\hat{b}_{\nu'}^\dagger)
          \sum_{s} v_{s \nu} v_{s \nu'}
          \nonumber \\
          &+\frac{1}{2} \sum_{\nu \nu'}
          (\hat{b}_\nu\hat{b}_{\nu'} + \hat{b}_\nu\hat{b}_{\nu'}^\dagger + \hat{b}_\nu^\dagger\hat{b}_{\nu'} + \hat{b}_\nu^\dagger\hat{b}_{\nu'}^\dagger)
          \frac{\omega_{\nu'}^2}{(\omega_\nu \omega_{\nu'})^{1/2}}
          \sum_{s } v_{s \nu} v_{s \nu'} 
          \nonumber \\
          &=\frac{1}{2} \sum_{\nu} \omega_\nu (\hat{b}_\nu\hat{b}_\nu^\dagger + \hat{b}_\nu^\dagger\hat{b}_\nu)
          =\sum_{\nu} \omega_\nu \left(\hat{b}_\nu^\dagger\hat{b}_\nu + \frac{1}{2}\right)
         ,
\end{align}

\section{Periodic Coupled Harmonic Oscillator}

The hamiltonian operator is
\begin{align}
    \hat{H} = \frac{1}{2}\sum_{\textbf{T}s\alpha}\frac{\hat{p}_{\textbf{T}s\alpha}^2}{M_s} 
    + \frac{1}{2}\sum_{\textbf{T}s\alpha \textbf{T}'s'\alpha'} \hat{u}_{\textbf{T}s\alpha} 
    C_{\textbf{T}s\alpha \textbf{T}'s'\alpha'} \hat{u}_{\textbf{T}'s'\alpha'},
\end{align}
where $\hat{u}_{\textbf{T}s\alpha}$ and $\hat{p}_{\textbf{T}s\alpha}$ are the position and the momentum operator.
These operators satisfies the following commutation relation:
\begin{align}
    [\hat{u}_{\textbf{T}s\alpha}, \hat{p}_{\textbf{T}'s'\alpha'}] \equiv i \delta_{\textbf{T}\textbf{T}'}\delta_{s s'}\delta_{\alpha \alpha'}.
\end{align}
The Fourier-transformed operators are defined as follows:
\begin{align}
    \hat{U}_{\textbf{q}s\alpha} \equiv \frac{1}{N_C^{1/2}} \sum_{\textbf{T}} \hat{u}_{\textbf{T}s\alpha} e^{i\textbf{q}\cdot\textbf{T}}
    ,
    \hat{P}_{\textbf{q}s\alpha} \equiv \frac{1}{N_C^{1/2}} \sum_{\textbf{T}} \hat{p}_{\textbf{T}s\alpha} e^{i\textbf{q}\cdot\textbf{T}}
    \\
    \hat{u}_{\textbf{T}s\alpha} = \frac{1}{N_C^{1/2}} \sum_{\textbf{q}} \hat{U}_{\textbf{q}s\alpha} e^{-i\textbf{q}\cdot\textbf{T}}
    ,
    \hat{p}_{\textbf{T}s\alpha} = \frac{1}{N_C^{1/2}} \sum_{\textbf{q}} \hat{P}_{\textbf{q}s\alpha} e^{-i\textbf{q}\cdot\textbf{T}},
    \label{eq_ftup}
\end{align}
where $N_C$ is the number of cells within the Born–-von Karman boundary condition.
They also satisfy the commutation relation.
\begin{align}
    [\hat{U}_{\textbf{q}s\alpha}, \hat{P}_{\textbf{q}'s'\alpha'}^\dagger] 
    &= \frac{1}{N_C}\sum_{\textbf{T}\textbf{T}'} e^{i\textbf{q}\cdot\textbf{T}}e^{-i\textbf{q}'\cdot\textbf{T}'}
    [\hat{u}_{\textbf{T}s\alpha}, \hat{p}_{\textbf{T}'s'\alpha'}]
    =
    i \frac{1}{N_C}\sum_{\textbf{T}} e^{i(\textbf{q}-\textbf{q}')\cdot\textbf{T}}
    \delta_{s s'}\delta_{\alpha \alpha'}
    \\
    &=i \delta_{\textbf{q}\textbf{q}'}\delta_{s s'}\delta_{\alpha \alpha'} .
\end{align}
The hamiltonian becomes
\begin{align}
    \hat{H} &= \frac{1}{2 N_C}\sum_{\textbf{q}\textbf{q}'\textbf{T}s\alpha}
    \frac{\hat{P}_{\textbf{q}s\alpha}\hat{P}_{\textbf{q}'s\alpha} e^{i(\textbf{q}+\textbf{q}')\cdot\textbf{T}}}{M_s} 
    + \frac{1}{2 N_C}\sum_{\textbf{q}\textbf{q}'s\alpha s'\alpha'}
    \hat{U}_{\textbf{q}s\alpha}  \hat{U}_{\textbf{q}'s'\alpha'}
    \sum_{\textbf{T} \textbf{T}'}
    C_{\textbf{T}s\alpha \textbf{T}'s'\alpha'}
    e^{i\textbf{q}\cdot\textbf{T}}e^{i\textbf{q}'\cdot\textbf{T}'}
    \nonumber \\
    &=
    \frac{1}{2}\sum_{\textbf{q}\textbf{q}'\textbf{T}s\alpha}
    \frac{\hat{P}_{\textbf{q}s\alpha}^\dagger \hat{P}_{\textbf{q}s\alpha}}{M_s} 
    + \frac{1}{2 N_C}\sum_{\textbf{q}\textbf{q}'s\alpha s'\alpha'}
    \hat{U}_{\textbf{q}s\alpha}  \hat{U}_{\textbf{q}'s'\alpha'}
    \sum_{\textbf{T} \textbf{T}'}
    C_{\textbf{0}s\alpha (\textbf{T}'-\textbf{T})s'\alpha'}
    e^{i\textbf{q}'\cdot(\textbf{T}'-\textbf{T})}e^{i(\textbf{q}+\textbf{q}')\cdot\textbf{T}}
    \nonumber \\
    &=
    \sum_{\textbf{q}} \left(
    \frac{1}{2}\sum_{s\alpha}
    \frac{\hat{P}_{\textbf{q}s\alpha}^\dagger \hat{P}_{\textbf{q}s\alpha}}{M_s} 
    + \frac{1}{2}\sum_{s\alpha s'\alpha'}
    \hat{U}_{\textbf{q}s\alpha}^\dagger \tilde{C}_{\textbf{q}s\alpha s'\alpha'} \hat{U}_{\textbf{q}s'\alpha'}
    \right),
\end{align}
where
\begin{align}
    \tilde{C}_{\textbf{q}s\alpha s'\alpha'} \equiv
    \sum_{\textbf{T}}
    C_{\textbf{0}s\alpha \textbf{T}s'\alpha'}
    e^{i\textbf{q}\cdot\textbf{T}}
\end{align}
With the same discussion in the previous section, we obtain the following results:
\begin{align}
    \hat{U}_{\textbf{q}s\alpha} &\equiv \sum_{\nu} \frac{v_{s\alpha \textbf{q}\nu}}{(M_s \omega_{\textbf{q}\nu})^{1/2}} 
    (\hat{b}_{\textbf{q}\nu} + \hat{b}_{\textbf{q}\nu}^\dagger),
    \label{eq_ub}
    \\
    \hat{P}_{\textbf{q}s\alpha} &\equiv \sum_{\nu} (M_s \omega_{\textbf{q}\nu})^{1/2} v_{s\alpha \textbf{q}\nu}
    (-i \hat{b}_{\textbf{q}\nu} + i \hat{b}_{\textbf{q}\nu}^\dagger),
    \\
    \sum_{s'\alpha'} \frac{\tilde{C}_{\textbf{q}s\alpha s'\alpha'}}{(M_s M_{s'})^{1/2}} v_{s'\alpha' \textbf{q}\nu} 
    &= \omega_{\textbf{q}\nu}^2 v_{s\alpha \textbf{q}\nu}
    \label{eigen}
    \\
    \hat{H} &=\sum_{{\textbf{q}\nu}} \omega_{\textbf{q}\nu} \left(\hat{b}_{\textbf{q}\nu}^\dagger\hat{b}_{\textbf{q}\nu} + \frac{1}{2}\right)
    \\
    [\hat{U}_{\textbf{q}s\alpha}, \hat{P}_{\textbf{q}'s'\alpha'}^\dagger] 
    &= \sum_{\nu \nu'} v_{s \alpha \textbf{q} \nu} v_{s' \alpha' \textbf{q}' \nu'}^*
    \frac{i}{2}(-[\hat{b}_{\textbf{q} \nu}, \hat{b}_{\textbf{q}' \nu'}] + [\hat{b}_{\textbf{q} \nu}, \hat{b}_{\textbf{q}' \nu'}^\dagger]
    -[\hat{b}_{\textbf{q} \nu}^\dagger, \hat{b}_{\textbf{q}' \nu'}] + [\hat{b}_{\textbf{q} \nu}^\dagger, \hat{b}_{\textbf{q}' \nu'}^\dagger])
    \nonumber \\
    &= i \delta_{\textbf{q}\textbf{q}'} \sum_\nu v_{s \alpha \textbf{q} \nu} v_{s' \alpha' \textbf{q}' \nu}^* 
    = i \delta_{\textbf{q}\textbf{q}'} \delta_{s s'}\delta_{\alpha \alpha'},
\end{align}

\section{Electron-phonon vertex}

Electron-nuclear Hamiltonian in 2nd quantization representation

\begin{align}
    \hat{H}_{en} = \sum_{\sigma} \int d^3 r \hat{\psi}_{\sigma}(\textbf{r})^\dagger 
    \hat{\psi}_{\sigma}(\textbf{r}) V(\textbf{r};\{\hat{R}_{\textbf{T}s\alpha}\}),
\end{align}
where $\hat{R}_{\textbf{T}s\alpha}$ is the position operator of nuclear as
\begin{align}
    \hat{R}_{\textbf{T}s\alpha} = R^0_{\textbf{T}s\alpha} + \hat{u}_{\textbf{T}s\alpha},
\end{align}
$R^0_{\textbf{T}s\alpha}$ is equiribrium position.

We expand $V(\textbf{r};\{\hat{R}_{\textbf{T}s\alpha}\})$ arround $R^0_{\textbf{T}s\alpha}$
and obtain
\begin{align}
    \hat{H}_{en} = \sum_{\sigma} \int d^3 r \hat{\psi}_{\sigma}(\textbf{r})^\dagger 
    \hat{\psi}_{\sigma}(\textbf{r}) V(\textbf{r};\{R^0_{\textbf{T}s\alpha}\})
    +
    \sum_{\sigma} \int d^3 r \hat{\psi}_{\sigma}(\textbf{r})^\dagger 
    \hat{\psi}_{\sigma}(\textbf{r}) \sum_{\textbf{T}s\alpha} 
    \frac{\partial V(\textbf{r};\{R^0\})}{\partial R^0_{\textbf{T}s\alpha}}
    \hat{u}_{\textbf{T}s\alpha}.
\end{align}
The first term is the electron-fixed nuclear interaction term, and the second term is the electron-phonon interaction term $\hat{H}_{ep}$.
By expanding $\hat{\psi}_{\sigma}$ with Bloch orbitals $\varphi_{n\textbf{k}}$
\begin{align}
    \hat{\psi}_{\sigma}(\textbf{r}) = \sum{n\textbf{k}}\varphi_{n\textbf{k}}(\textbf{r})
    \hat{c}_{n\textbf{k}\sigma},
\end{align}
we obtain
\begin{align}
    \hat{H}_{ep} = \sum_{\sigma n n' \textbf{k} \textbf{k}' \textbf{T}s\alpha}
    \hat{c}_{n\textbf{k}\sigma}^\dagger \hat{c}_{n'\textbf{k}'\sigma}
    \hat{u}_{\textbf{T}s\alpha}
    \int d^3 r \varphi_{n\textbf{k}}^*(\textbf{r})\varphi_{n'\textbf{k}'}(\textbf{r})
    \frac{\partial V(\textbf{r};\{R^0\})}{\partial R^0_{\textbf{T}s\alpha}}.
\end{align}
By using Eqs (\ref{eq_ftup}, \ref{eq_ub}) and 
$\varphi_{n\textbf{k}}(\textbf{r}) \equiv N^{-1/2}_C e^{i \textbf{k} \cdot \textbf{r}} 
\chi_{n\textbf{k}}(\textbf{r})$, we obtain
\begin{align}
    \hat{H}_{ep} = \sum_{\sigma n n' \textbf{k} \textbf{k}' \textbf{q} \nu}
    \hat{c}_{n\textbf{k}\sigma}^\dagger \hat{c}_{n'\textbf{k}'\sigma}
    (\hat{b}_{\textbf{q}\nu} + \hat{b}_{\textbf{q}\nu}^\dagger)
    g_{n\textbf{k} n'\textbf{k}'}^{\textbf{q}\nu},
\end{align}
where $g_{n\textbf{k} n'\textbf{k}'}^{\textbf{q}\nu}$ is the electron-phonon vertex as
\begin{align}
    g_{n\textbf{k} n'\textbf{k}'}^{\textbf{q}\nu} &\equiv
    \sum_{\textbf{T}s\alpha}
    e^{-i\textbf{q}\cdot\textbf{T}} 
    \frac{v_{s\alpha \textbf{q}\nu}}{(N_C^3 M_s \omega_{\textbf{q}\nu})^{1/2}} 
    \int d^3 r e^{i(\textbf{k}'-\textbf{k}) \cdot \textbf{r}}
    \chi_{n\textbf{k}}^*(\textbf{r})\chi_{n'\textbf{k}'}(\textbf{r})
    \frac{\partial V(\textbf{r};\{R^0\})}{\partial R^0_{\textbf{T}s\alpha}}
    \nonumber \\
    &= 
    \sum_{\textbf{T} \textbf{T}'s\alpha}
    e^{-i\textbf{q}\cdot\textbf{T}} 
    \frac{v_{s\alpha \textbf{q}\nu}}{(N_C^3 M_s \omega_{\textbf{q}\nu})^{1/2}} 
    \int_{\textrm{cell}} d^3 r
    e^{i(\textbf{k}'-\textbf{k}) \cdot (\textbf{r}+\textbf{T}')}
    \chi_{n\textbf{k}}^*(\textbf{r})\chi_{n'\textbf{k}'}(\textbf{r})
    \frac{\partial V(\textbf{r}+\textbf{T}';\{R^0\})}{\partial R^0_{\textbf{T}s\alpha}}
    \nonumber \\
    &= 
    \sum_{\textbf{T}}
    e^{i(\textbf{k}'-\textbf{k}-\textbf{q})\cdot\textbf{T}} 
    \sum_{\textbf{T}'s\alpha}
    \frac{v_{s\alpha \textbf{q}\nu}}{(N_C^3 M_s \omega_{\textbf{q}\nu})^{1/2}} 
    \int_{\textrm{cell}} d^3 r
    e^{i(\textbf{k}'-\textbf{k}) \cdot (\textbf{r}+\textbf{T}'-\textbf{T})}
    \chi_{n\textbf{k}}^*(\textbf{r})\chi_{n'\textbf{k}'}(\textbf{r})
    \frac{\partial V(\textbf{r};\{R^0\})}{\partial R^0_{(\textbf{T}-\textbf{T}')s\alpha}}
    \nonumber \\
    &= 
    \delta_{\textbf{k}',\textbf{k}+\textbf{q}}
    \sum_{s\alpha}
    \frac{v_{s\alpha \textbf{q}\nu}}{(N_C M_s \omega_{\textbf{q}\nu})^{1/2}} 
    \int_{\textrm{cell}} d^3 r
    \chi_{n\textbf{k}}^*(\textbf{r})\chi_{n'\textbf{k}+\textbf{q}}(\textbf{r})
    \sum_{\textbf{T}}e^{i \textbf{q} \cdot (\textbf{r}-\textbf{T})}
    \frac{\partial V(\textbf{r};\{R^0\})}{\partial R^0_{\textbf{T}s\alpha}}.
\end{align}
If we use $V_{\textrm{KS}}$ alternative to $V$, we obtain the screened electron-phonon vertex.
The screened deformation potential $\sum_{\textbf{T}}e^{i \textbf{q} \cdot (\textbf{r}-\textbf{T})} \partial V_\textrm{KS}(\textbf{r};\{R^0\})/\partial R^0_{\textbf{T}s\alpha}$ is obtained as a biproduct of DFPT calculation.

This deformation potential has lattice periodicity as
\begin{align}
    \sum_{\textbf{T}}e^{i \textbf{q} \cdot (\textbf{r}+\textbf{T}'-\textbf{T})}
    \frac{\partial V(\textbf{r}+\textbf{T}';\{R^0\})}{\partial R^0_{\textbf{T}s\alpha}}
    =
    \sum_{\textbf{T}}e^{i \textbf{q} \cdot (\textbf{r}-\textbf{T})}
    \frac{\partial V(\textbf{r};\{R^0\})}{\partial R^0_{\textbf{T}s\alpha}}
\end{align}

\section{Density functional perturbation theory for lattice}

Force constant
\begin{align}
    C_{\textbf{T}s\alpha \textbf{T}'s'\alpha'} &\equiv 
    \frac{\partial^2 E}{\partial R^0_{\textbf{T}s\alpha} \partial R^0_{\textbf{T}'s'\alpha'}}
    = - \frac{\partial F_{\textbf{T}s\alpha}}{\partial R^0_{\textbf{T}'s'\alpha'}}
    = \frac{\partial}{\partial R^0_{\textbf{T}'s'\alpha'}}
    \left(-F_{\textbf{T}s\alpha}^\textrm{C} + \int d^3r \rho(\textbf{r}) 
    \frac{\partial V(\textbf{r}; \{R^0\}) }{\partial R^0_{\textbf{T}s\alpha}}\right)
    \nonumber \\
    &= 
    \frac{\partial^2 E_\textrm{C}}{\partial R^0_{\textbf{T}s\alpha} \partial R^0_{\textbf{T}'s'\alpha'}}
    +\int d^3r \rho(\textbf{r}) 
    \frac{\partial^2 V(\textbf{r}; \{R^0\}) }{\partial R^0_{\textbf{T}s\alpha} \partial R^0_{\textbf{T}'s'\alpha'}}
    +\int d^3r \frac{\partial \rho(\textbf{r})}{\partial R^0_{\textbf{T}'s'\alpha'}}
    \frac{\partial V(\textbf{r}; \{R^0\}) }{\partial R^0_{\textbf{T}s\alpha}}
\end{align}
Dynamical materix
\begin{align}
    \tilde{C}_{\textbf{q}s\alpha s'\alpha'} &\equiv
    \sum_{\textbf{T}}
    C_{\textbf{0}s\alpha \textbf{T}s'\alpha'}
    e^{i\textbf{q}\cdot\textbf{T}}
    \nonumber \\
    &=\sum_{\textbf{T}}
    e^{i\textbf{q}\cdot\textbf{T}}\left(
    \frac{\partial^2 E_\textrm{C}}{\partial R^0_{\textbf{}s\alpha} \partial R^0_{\textbf{T}'s'\alpha'}}
    +\int d^3r \rho(\textbf{r}) 
    \frac{\partial^2 V(\textbf{r}; \{R^0\}) }{\partial R^0_{\textbf{0}s\alpha} \partial R^0_{\textbf{T}s'\alpha'}}
    \right)
    +
    \int d^3r 
    \frac{\partial V(\textbf{r}; \{R^0\}) }{\partial R^0_{\textbf{0}s\alpha}}
    \sum_{\textbf{T}}
    e^{i\textbf{q}\cdot\textbf{T}}\frac{\partial \rho(\textbf{r})}{\partial R^0_{\textbf{T}s'\alpha'}}
\end{align}
Monochromatic perturbation
\begin{align}
    \sum_{\textbf{T}}
    e^{i\textbf{q}\cdot\textbf{T}}\frac{\partial \rho(\textbf{r})}{\partial R^0_{\textbf{T}s\alpha}}
    &=
    2 \sum_{\textbf{T}n\textbf{k}}e^{i\textbf{q}\cdot\textbf{T}} 
    \left( 
        \frac{\partial \varphi_{n\textbf{k}}^*(\textbf{r})}{\partial R^0_{\textbf{T}s\alpha}}\varphi_{n\textbf{k}}(\textbf{r})
       +\varphi_{n\textbf{k}}^*(\textbf{r})\frac{\partial \varphi_{n\textbf{k}}(\textbf{r})}{\partial R^0_{\textbf{T}s\alpha}}
    \right)
    \nonumber \\
    &=
    2 \sum_{\textbf{T}n\textbf{k}}e^{i\textbf{q}\cdot\textbf{T}} 
    \left( 
        \frac{\partial \chi_{n\textbf{k}}^*(\textbf{r})}{\partial R^0_{\textbf{T}s\alpha}}\chi_{n\textbf{k}}(\textbf{r})
       +\chi_{n\textbf{k}}^*(\textbf{r})\frac{\partial \chi_{n\textbf{k}}(\textbf{r})}{\partial R^0_{\textbf{T}s\alpha}}
    \right)
    \nonumber \\
    &=
    2 e^{i\textbf{q}\cdot\textbf{r}}\sum_{\textbf{T}n\textbf{k}}e^{i\textbf{q}\cdot(\textbf{T}-\textbf{r})} 
    \left( 
        \frac{\partial \chi_{n\textbf{k}}^*(\textbf{r})}{\partial R^0_{\textbf{T}s\alpha}}\chi_{n\textbf{k}}(\textbf{r})
       +\chi_{n\textbf{k}}^*(\textbf{r})\frac{\partial \chi_{n\textbf{k}}(\textbf{r})}{\partial R^0_{\textbf{T}s\alpha}}
    \right)
\end{align}

\begin{align}
    \left(-\frac{\nabla^2}{2}+V_\textrm{KS}(\textbf{r}) - \varepsilon_{n\textbf{k}} \right)
    \frac{\partial \varphi_{n\textbf{k}}(\textbf{r})}{\partial R^0_{\textbf{T}s\alpha}}
    &=
    \left(\frac{\partial \varepsilon_{n\textbf{k}}}{\partial R^0_{\textbf{T}s\alpha}}
    -\frac{\partial V_\textrm{KS}(\textbf{r})}{\partial R^0_{\textbf{T}s\alpha}}\right)
    \varphi_{n\textbf{k}}(\textbf{r})
    \nonumber \\
    \left(-\frac{(i\textbf{k}+\nabla)^2}{2}+V_\textrm{KS}(\textbf{r}) - \varepsilon_{n\textbf{k}} \right)
    \frac{\partial \chi_{n\textbf{k}}(\textbf{r})}{\partial R^0_{\textbf{T}s\alpha}}
    &=
    \left(\frac{\partial \varepsilon_{n\textbf{k}}}{\partial R^0_{\textbf{0}s\alpha}}
    -\frac{\partial V_\textrm{KS}(\textbf{r})}{\partial R^0_{\textbf{T}s\alpha}}\right)
    \chi_{n\textbf{k}}(\textbf{r})
    \nonumber \\
    \sum_{\textbf{T}}e^{i\textbf{q}\cdot(\textbf{T}-\textbf{r})} 
    \left(-\frac{(i\textbf{k}+\nabla)^2}{2}+V_\textrm{KS}(\textbf{r}) - \varepsilon_{n\textbf{k}} \right)
    \frac{\partial \chi_{n\textbf{k}}(\textbf{r})}{\partial R^0_{\textbf{T}s\alpha}}
    &=
    \sum_{\textbf{T}}e^{i\textbf{q}\cdot(\textbf{T}-\textbf{r})} 
    \left(\frac{\partial \varepsilon_{n\textbf{k}}}{\partial R^0_{\textbf{0}s\alpha}}
    -\frac{\partial V_\textrm{KS}(\textbf{r})}{\partial R^0_{\textbf{T}s\alpha}}\right)
    \chi_{n\textbf{k}}(\textbf{r})
    \nonumber \\
    \left(-\frac{(i\textbf{k}+i\textbf{q}+\nabla)^2}{2}+V_\textrm{KS}(\textbf{r}) - \varepsilon_{n\textbf{k}} \right)
    \sum_{\textbf{T}}e^{i\textbf{q}\cdot(\textbf{T}-\textbf{r})} 
    \frac{\partial \chi_{n\textbf{k}}(\textbf{r})}{\partial R^0_{\textbf{T}s\alpha}}
    &=
    \left(N_C \delta_{\textbf{q}\textbf{0}}\frac{\partial \varepsilon_{n\textbf{k}}}{\partial R^0_{\textbf{0}s\alpha}}
    -\sum_{\textbf{T}}e^{i\textbf{q}\cdot(\textbf{T}-\textbf{r})} 
    \frac{\partial V_\textrm{KS}(\textbf{r})}{\partial R^0_{\textbf{T}s\alpha}}\right)
    \chi_{n\textbf{k}}(\textbf{r}).
\end{align}
Each component has lattice periodicity
\begin{align}
    \sum_{\textbf{T}}e^{i\textbf{q}\cdot(\textbf{T}-\textbf{r}-\textbf{T}')} 
    \frac{\partial \chi_{n\textbf{k}}(\textbf{r}+\textbf{T}')}{\partial R^0_{\textbf{T}s\alpha}}
    &=
    \sum_{\textbf{T}}e^{i\textbf{q}\cdot(\textbf{T}-\textbf{r})} 
    \frac{\partial \chi_{n\textbf{k}}(\textbf{r})}{\partial R^0_{\textbf{T}s\alpha}}
    \\
    \sum_{\textbf{T}}e^{i\textbf{q}\cdot(\textbf{T}-\textbf{r}-\textbf{T}')} 
    \frac{\partial V_\textrm{KS}(\textbf{r}+\textbf{T}')}{\partial R^0_{\textbf{T}s\alpha}}
    &=
    \sum_{\textbf{T}}e^{i\textbf{q}\cdot(\textbf{T}-\textbf{r})} 
    \frac{\partial V_\textrm{KS}(\textbf{r})}{\partial R^0_{\textbf{T}s\alpha}}
\end{align}
Deformation potential is computed as follows:
\begin{align}
    \sum_{\textbf{T}}e^{i\textbf{q}\cdot(\textbf{T}-\textbf{r})} 
    \frac{\partial V_\textrm{KS}(\textbf{r})}{\partial R^0_{\textbf{T}s\alpha}}   
    &=
    \sum_{\textbf{T}}e^{i\textbf{q}\cdot(\textbf{T}-\textbf{r})} 
    \left\{
    \frac{\partial V(\textbf{r})}{\partial R^0_{\textbf{T}s\alpha}}
    +
    \int d^3r'
    \left(
    \frac{\delta V_\textrm{H}(\textbf{r})}{\delta \rho(\textbf{r}')}
    +\frac{\delta V_{XC}(\textbf{r})}{\delta \rho(\textbf{r}')}
    \right)
    \frac{\partial \rho(\textbf{r}')}{\partial R^0_{\textbf{T}s\alpha}}
    \right\}
    \nonumber \\
    &=
    \sum_{\textbf{T}}e^{i\textbf{q}\cdot(\textbf{T}-\textbf{r})} 
    \left\{
    \frac{\partial V(\textbf{r})}{\partial R^0_{\textbf{T}s\alpha}}
    +
    \int d^3r'
    \left(
    \frac{1}{|\textbf{r}-\textbf{r}'|}
    + f_{XC}(\textbf{r}, \textbf{r}')
    \right)
    \frac{\partial \rho(\textbf{r}')}{\partial R^0_{\textbf{T}s\alpha}}
    \right\}
    \nonumber \\
    &=
    \sum_{\textbf{T}}e^{i\textbf{q}\cdot(\textbf{T}-\textbf{r})} 
    \frac{\partial V(\textbf{r})}{\partial R^0_{\textbf{T}s\alpha}}
    +
    \int d^3r'
    e^{i\textbf{q}\cdot(\textbf{r}'-\textbf{r})} 
    \left(
    \frac{1}{|\textbf{r}-\textbf{r}'|}
    + f_{XC}(\textbf{r}, \textbf{r}')
    \right)
    \sum_{\textbf{T}}e^{i\textbf{q}\cdot(\textbf{T}-\textbf{r}')} 
    \frac{\partial \rho(\textbf{r}')}{\partial R^0_{\textbf{T}s\alpha}}
\end{align}

\end{document}